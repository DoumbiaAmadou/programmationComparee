\documentclass[a4paper, 12pt]{article}

\usepackage[french]{babel}
\usepackage[T1]{fontenc}
\usepackage[margin=2.5cm]{geometry}
\usepackage[utf8]{inputenc}
\usepackage{mathptmx}

\begin{document}
\section*{Généralités}
Le projet est écrit en OCaml et utilise les bibliothèques suivantes :

\begin{itemize}
\item oCurl pour avoir un \textit{binding} sur la \texttt{libcurl} ;
\item YoJson pour la lecture et l'écriture d'objets JSON ;
\end{itemize}

\section*{Architecture}
Le projet s'articule autour de modules OCaml qui exposent chacun une
interface de telle sorte que chaque module est indépendant.
\end{document}
